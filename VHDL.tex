%%%%%%%%%%%%%%%%%%%%%%%%%%%%%%%%%%%%%%%%%%%%%%%%%%%%%%%%%%%%%%%%%%%%%%%%%%%%%%%%%%%%%%
% VHDL
%%%%%%%%%%%%%%%%%%%%%%%%%%%%%%%%%%%%%%%%%%%%%%%%%%%%%%%%%%%%%%%%%%%%%%%%%%%%%%%%%%%%%%
\section{VHDL}

\begin{minipage}{0.45\textwidth}
	\begin{tabular}{l}
		\textbf{Überblick}\\
		Kreatives erkennen $\rightarrow$ Mensch\\
		Fleissarbeit $\rightarrow$ CAD Systeme {\tiny (Computer Aided Design)}\\
		"Verilog"' und "'SystemC"' sind alternativen {\tiny (weniger verbreitet)}\\
		Vorteile von VHDL:\\
		\quad $\cdot$ Fokus auf Beschreibung des Verhaltens \\
		\quad $\cdot$ geprägt von kreativen, erfahrungsbasierten Design \\
		\quad $\cdot$ Realisierung von wiederverwendbaren Blöcken \\
	\end{tabular}
\end{minipage}
\begin{minipage}{0.5\textwidth}
	\begin{tabular}{l}
		\textbf{Geschichte}\\
		1980: DoD {\tiny (US Department od Defence)} beauftragt IBM, Intermetrics \\
		\quad und TI eine gemeinsame Hardwaresprache zu definieren\\
		Namen setzt sich von VHSIC {\tiny (Very High Speed Integrated Circuits)} des Dod\\
		\quad und HDL {\tiny (Hardware Description Languege)} zusammen \\
		1987: DoD verzichtet auf das Exklusivrecht \\
		\quad $\rightarrow$ IEEE führt den STD weiter {\tiny (neuster: IEEE 1076.1)} \\
	\end{tabular}
\end{minipage}

\begin{minipage}{0.45\textwidth}
	\begin{tabular}{l}
		\textbf{Charakteristische Elemente}\\
		$\cdot$ Hierarchie und Konnektivität \\
		$\cdot$ Beschreibung und Interpretation \textbf{paralleler} Abläufe \\
		$\cdot$ Beschreibung elektrischer Signale \\
		$\cdot$ Beschreibung des Zeitlichen Verhaltens  \\
		$\cdot$ Parametisierbarkeit von Modellen \\
	\end{tabular}
\end{minipage}
\begin{minipage}{0.32\textwidth}
	\begin{tabular}{l}
		\textbf{Eigenschaften}\\
		$\cdot$ nicht "case sensitiv"  {\tiny Gross- Kleinschreibung}\\
		$\cdot$ Identifier alphanumerisch  \\
		\quad$\cdot$ immer mit Buchstabe beginnen \\
		\quad$\cdot$ keine "\_"' am Ende oder doppelt  \\
		\quad$\cdot$ keine Schlüsselwörter verwenden \\
	\end{tabular}
\end{minipage}
\begin{minipage}{0.2\textwidth}
	\begin{VHDL}
<>  -- identifier
[]  -- Optional Element
{}  -- kann beliebig oft wiederholt werden
	\end{VHDL}
\end{minipage}