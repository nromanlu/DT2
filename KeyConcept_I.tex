%%%%%%%%%%%%%%%%%%%%%%%%%%%%%%%%%%%%%%%%%%%%%%%%%%%%%%%%%%%%%%%%%%%%%%%%%%%%%%%%%%%%%%
% Key Concept I        
%%%%%%%%%%%%%%%%%%%%%%%%%%%%%%%%%%%%%%%%%%%%%%%%%%%%%%%%%%%%%%%%%%%%%%%%%%%%%%%%%%%%%%
\section{Key Concept I \tiny Hierarchie und konektivität}

\subsection{Library}

\begin{minipage}{0.519\textwidth}
	\begin{tabular}{l l}
		\multicolumn{2}{l}{$\cdot$ bereits kompiliert} \\
		\multicolumn{2}{l}{$\cdot$ kann Komponente und/oder Packages enthalten} \\
		$\cdot$ work {\tiny Arbeitsbibliothek} 		& : immer vorhanden {\tiny(wird automatisch generiert)}\\
		$\cdot$ std {\tiny Standartfunktionen} 		& : im IEEE 1076 spezifiziert {\tiny(BIT, INTEGER, ...)}\\
		$\cdot$ ieee {\tiny Praktische Packages}	& : z.B. std\_logic \\
		$\cdot$ PLD, ... {\tiny Hersteller Lib.} 	& : Beschreibung spezifischer Elemente \\
\end{tabular}
\end{minipage}
\begin{minipage}{0.48\textwidth}
	\begin{VHDL}
-- deklaration der Bibliotek
library ieee;	
-- Konstante aus Bibliothek lesen
<variable>:=<library_name>.<consr_name>;
-- Bibliothek sichtbar machen
use <library_name>.<package_name>.<element_name>; -- oder
use <library_name>.all;
	\end{VHDL}
\end{minipage}

\newpage
%--------------------------------------------------------------------------------------
\subsection{Entity}

\begin{minipage}{0.519\textwidth}
	\begin{tabular}{l l}
		$\cdot$ Port 	& : alle digitale Signale, die von aussen sichtbar sind\\
		$\cdot$ Mode	& : in oder out - Ein- Ausgangssignale\\
						& . buffer - zurückführende Ausgangssignale\\
						& . inout - bidirektionale Signale z.B. Busse\\
		$\cdot$ Type 	& : bit, bit\_vector, std\_logic, ... \\
\end{tabular}
\end{minipage}
\begin{minipage}{0.48\textwidth}
	\begin{VHDL}
-- Syntax
entity <ENTITY_NAME> is
	port (
		{<PORT_NAME>: <mode> <type>;}
	);
end <ENTITY_NAME>;	
	\end{VHDL}
\end{minipage}
%--------------------------------------------------------------------------------------
\subsection{Architecture}

\begin{minipage}{0.519\textwidth}
	\begin{tabular}{l l}
		\multicolumn{2}{l}{	mögliche architecture\_names:} \\
		$\cdot$ Behavioral 	& : Verhaltensbeschreibung {\tiny(ähnlich wie Prog.sprachen)}\\
		$\cdot$ Structural	& : Strukturbeschreibung {\tiny(Schaltschema beschrieben)}\\
		$\cdot$ TB 			& : Test-Bench \\
		& \\
		& \\
\end{tabular}


\textbf{Beispiel für Structural}

\end{minipage}
\begin{minipage}{0.48\textwidth}
	\begin{VHDL}
-- Syntax
architecture <architecture_name> of <entity_name> is
	[Type_,Subtype_,Constant_,Signal_,Component_declaration]
begin
	{architecture_body: concurrent actions}
end [architecture_name];	
	\end{VHDL}
\end{minipage}

\begin{minipage}{0.01\textwidth}
	\text{ } %platzhalter
\end{minipage}
\begin{minipage}{0.48\textwidth}
	\begin{VHDL}
-- Deklaration
component <component_name>
	port (
		{port_name: <port_mode><port_type>});
end component;

signal <internal_signal_name>: <sygnal_type>;	
	\end{VHDL}
\end{minipage}
\begin{minipage}{0.02\textwidth}
	\text{ } %platzhalter
\end{minipage}
\begin{minipage}{0.48\textwidth}
	\begin{VHDL}
-- Initialisierung
<instance_name>: <component_name>
	port map (	-- explizit
		<instance_port_name> => <external_signal_name>,
		<instance_port2_name> => <external_signal2_name>);	
		-- or implizit
		(<external_signal_name>, <external_signal2_name>)
	\end{VHDL}
\end{minipage}